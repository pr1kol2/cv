\documentclass[a4paper, 12pt, sans]{moderncv}

\usepackage[a4paper, top=0.5cm, bottom=1.5cm, left=1.75cm, right=1.75cm]{geometry}
\usepackage[T1]{fontenc}
\usepackage[utf8]{inputenc}

\usepackage[english]{babel}

\usepackage{mathtext}
\usepackage{mathtools}
\usepackage{mathrsfs}
\usepackage{amsmath,amssymb}
\usepackage{float}

\usepackage{indentfirst}
\usepackage{longtable}


\moderncvstyle{classic}
\moderncvcolor{burgundy}
\setlength{\hintscolumnwidth}{2.1cm}

\firstname{Kirill}
\familyname{Grinko}
\mobile{+7 916 825 87 57}
\email{k.a.grinko@gmail.com}
\social[telegram]{pr1kol2}
\social[github]{pr1kol2005}

\begin{document}

\makecvtitle


\section{Personal}

\cvitem{Hard skills}{C++, Algorithms \&  data structures, Concurrency, C, Assembly x86 \& ARM, Python, LaTeX, Git, Bash, Docker, CMake, GoogleTest, Gitlab CI/CD, Qt, SFML.}

\cvitem{Soft skills}{Quick-learning, Hard-working, Organised, Outgoing and collaborative.}

\cvitem{Languages}{English (B2), Chinese (A1), Russian (native speaker).}

\cvitem{Hobbies}{Calisthenics, skiing, cycling.}


\section{Experience}

\cventry{Oct 2025 -- Present}{Software Engineering Intern at VK}{Database team}{}{}{}

\cventry{Spring 2025}{Concurrency course at MIPT}{}{}{}{Implemented various synchronization primitives using atomic operations only. Built a thread pool and stackful coroutines. Combined them to create fibers (user-space cooperative threads) and implemented synchronization primitives for them. Developed functional combinators for working with futures (representing values computed by asynchronous operations). Implemented a lock-free data structures (atomic shared\_ptr, stack, queue) using the hazard pointers scheme.}

\cventry{Fall 2024 -- Spring 2025}{C++ course at MIPT}{\href{https://github.com/pr1kol2005/cpp-3-4-term}{GitHub repo}}{}{}{Implemented template allocator-aware data structures (unordered\_map, list, smart pointers, strategy-based array, matrix), type-erased configuration system with vtable, compile-time 8-puzzle solver, JSON converter, geometry primitives, big\_integer.}

\cventry{Fall 2024 -- Spring 2025}{Algorithms and data structures course at MIPT}{\href{https://github.com/pr1kol2005/algorithms-collection.git}{GitHub repo}}{}{}{Implemented solutions to competitive programming problems covering fundamental algorithms and data structures, dynamic programming techniques, graph algorithms, algorithms on strings, and number theory algorithms.}

\section{Projects}

\cventry{June 2025}{Metrics lib}{\href{https://github.com/pr1kol2005/metrics_lib}{GitHub repo}}{C++, CMake, Bash}{}{A high-performance C++ library for collecting, aggregating, and writing metrics to a file. Uses lock-free containers implemented with a hazard pointer scheme for safe memory reclamation. Features an extensible architecture based on templates and interfaces, allowing users to easily define custom metrics. Includes a modular codebase with CI powered by GitHub Actions for automated building, formatting, and testing.}

\cventry{Fall 2024}{Graphing calculator}{\href{https://github.com/pr1kol2005/graphing-calculator}{GitHub repo}}{C++, SFML, CMake}{}{A graphing calculator and plotter application. The Bridge pattern is used to separate math logic from rendering.}

\cventry{Spring 2024}{Box with molecules}{\href{https://github.com/pr1kol2005/box-with-molecules}{GitHub repo}}{C++, Qt, CMake, Python}{}{A simulation of an ideal gas in an enclosed space, including a small research component to test the validity of the Maxwell distribution.}

\cventry{Fall 2023 -- Spring 2024}{Physics laboratory works}{\href{https://github.com/pr1kol2005/labs-mipt.git}{GitHub repo}}{LaTeX, Python}{}{A collection of completed laboratory works in physics, including theoretical calculations, experimental data analysis, and visualizations using Python with numpy and matplotlib.}

\newpage

\section{Education}

\cventry{2023 -- Present}{Moscow Institute of Physics and Technology}{finished 4th semester bachelor}{Overall GPA: 4.70/5; Programming courses GPA: 4.81/5}{}{System Programming and Applied Mathematics, Phystech School of Applied Mathematics and Informatics.}

\cventry{2019 -- 2023}{Moscow State School 57}{grades 8~-- 11}{}{GPA: 5/5}{Focus on physics and math. Graduated with federal and Moscow gold medals.}


\section{Achievements}

\cventry{2023}{All-Russian Olympiad in physics}{final stage participant, top 80 in country}{}{}{}

\cventry{2023}{Phystech (MIPT) Olympiad in physics}{final stage gold medal}{}{}{}

\cventry{2022}{Rosatom Olympiad in physics and maths}{final stage gold and silver medals}{}{}{}

\cventry{2019}{International Experimental Physics Olympiad}{bronze medal}{}{}{}

\cventry{2019}{Maxwell Physics Olympiad}{final stage silver medal}{}{}{}


\section{Extracurricular activities}

\cventry{2019 -- 2023}{Olympiad Physics Classes}{}{}{}{Theoretical and experimental training for All-Russian Olympiad for schoolchildren in physics, organized by the Moscow City Department of Education.}

\cventry{2020 -- 2022}{Yandex Lyceum}{}{}{}{Python programming classes for high school students. \href{https://academy.yandex.ru/lyceum/}{\emph{More info}}.}

\cventry{2021}{QuSoft Quantum Quest}{}{}{}{An online course on quantum computing for high school students, developed by Michael Walter and Māris Ozols. \href{https://www.quantum-quest.org}{\emph{More info}}.}


\end{document}